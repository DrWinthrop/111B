\documentclass{../signatures}
\labacronym{COM}
\labtitle{Compton Scattering}

\begin{document}

\maketitle

\names

\prelab

\begin{enumerate}

    \item What is Compton scattering? How does the Compton formula relate to a laboratory experiment?
    
    \item To get an idea of the maximum Compton energy shift for the 59.54 keV photons from the 241Am source, calculate the energy shift for a back-scattered photon off of a free electron.

    \item Calculate the energy shift for a back-scattered photon off of an aluminum nucleus. How does this compare to the shift from scattering off of an electron?

    \item What is a scattering cross-section? How does the Klein-Nishina formula relate to a laboratory experiment?

    \item Use the (classical) Thomson total cross section for scattering to estimate the probability of scattering off of a nucleus compared to the probability of scattering off of an electron. [See Melissinos for information on the Thomson cross-section and cross sections in general.]

    \item By what means does the CdTe detector detect photons? (Photoelectric effect? Compton scattering? Pair production? Bremstrahlung?)
       \\[36pt]
\end{enumerate}

\prelabsignatures

\midlab

\begin{enumerate}

    \item On day 5 of this lab, you should have produced a plot of scattered peak energies versus scattering angle, and made an estimate of the electron mass. Show them to a GSI and ask for a signature.
\\[36pt]
\end{enumerate}

\midlabsignatures{5}

\checkpointsection 

\begin{enumerate}

\item \checkp{Optimal Energy Range}

\item \checkp{Peak Finding}

\item \checkp{Additional Questions}

\item \checkp{Zeeman Picture}

\item \checkp{Zeeman Splitting}

\end{enumerate}

\end{document}