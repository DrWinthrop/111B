\documentclass{../signatures}
\labacronym{BRA}
\labtitle{Beta Ray Spectroscopy}

\begin{document}

\maketitle

\names

\prelab

\textbf{Watch the Beta Ray \& Error Analysis video with these questions at hand.}

\begin{enumerate}

    \item What are Beta Rays? How are they produced?
    
    \item How is radioactive decay related to nuclear binding energy? 

    \item What is the half-life of Cs137 and what are the decay modes of this nucleus? Sketch the two beta-decay distributions of Cs137. Qualitatively explain their general shapes and relative sizes. Indicate the maximum energy of each distribution.

    \item What is the "internal conversion process" and how does atomic electronic binding play a role in the electron's final kinetic energy?

    \item What is a Fermi-Kurie plot, and why do you need to use it?

    \item What does the Beta-ray spectrometer measure, and how does it measure it?

    \item Identify and describe the operation of each of the pieces of experimental apparatus: source, coils, preamp, amp, SCA, etc.
       \\[36pt]
\end{enumerate}

\prelabsignatures

\midlab

\begin{enumerate}

    \item On day 2 of this lab, you should have taken several quick spectra and determined an appropriate setting of the discriminator’s baseline parameter. Demonstrate and show the spectra to an instructor and ask for a signature. 
\\[36pt]
\end{enumerate}

\midlabsignatures{2}

\checkpointsection 

\begin{enumerate}

\item \checkp{LabView}

\item \checkp{Scope Image}

\item \checkp{Statistical Fluctuations}

\item \checkp{Hysteresis}

\item \checkp{Combined Fermi-Kurie Plot}

\end{enumerate}

\end{document}