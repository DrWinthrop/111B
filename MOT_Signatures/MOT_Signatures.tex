\documentclass{../signatures}
\labacronym{MOT}
\labtitle{Atom Trapping}

\begin{document}

\maketitle

\names

\prelab

\begin{enumerate}

    \item Describe quantitatively how closed-loop control allows one to steer the output of a system to a desired level and to make the system immune to many disturbances.
    
    \item How is the emission frequency of the laser measured and controlled in this experiment?

    \item How does a half-wave plate affect the polarization of an incident, linear polarized laser beam, and how is this effect used in your experiment. How does a quarter-wave plate affect the polarization of an incident circular polarized laser beam?

    \item Laser light with wavelength near 780 nm sent through a room-temperature vapor of rubidium may be attenuated by the vapor. If one records the transmitted laser power as one varies the laser frequency over a broad range (many GHz), what features does one expect to see (positions, widths, signal strengths) and what is their origin?

    \item What are the safety \underline{requirements} for working with this laser?
       \\[36pt]
\end{enumerate}

\prelabsignatures

\midlab
On day 3 of this lab, you should have measured the frequency response of the system, created a Bode plot of your measurements, and successfully produced a stable MOT. Show them to an instructor and ask for a signature.
\begin{enumerate}
    \item Explain how the Doppler shift yields a damping force when atoms are exposed to counter-propagating laser beams of equal frequency. Derive the damping coefficient $\beta$.
    
    \item Referencing the section in the laboratory manual on the ``release-and-catch'' method of measuring the temperature, derive an expression for the radius $r$ of the atomic gas as a function of the time of flight tTOF.
\\[36pt]
\end{enumerate}

\midlabsignatures{3}

\checkpointsection
\begin{enumerate}

\item \checkp{Power}

\item \checkp{Two Quarter-Wave Plates}

\item \checkp{Four Peaks}

\item \checkp{Adjustments}

\item \checkp{Results and Steps}

\item \checkp{Methods of Measuring Temperature}

\end{enumerate}

\end{document}