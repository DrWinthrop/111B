\documentclass{../signatures}
\labacronym{ATM}
\labtitle{Atomic Physics}

\begin{document}

\maketitle

\names

\prelab

\begin{enumerate}

    \item Draw an energy level diagram for hydrogen. Show the transitions that produce the Balmer series. What is the formula that gives the wavelengths of these lines for the simple Bohr hydrogen atom? How does the derivation of the energy levels in the Bohr model differ from the quantum mechanical methods?
    
    \item Write the following atomic state's abbreviation in "full spectroscopic notation" (the notation for states written in the \emph{n,l,j,m$_j$} basis, described in the first instructional video). This \href{http://www.physics.byu.edu/faculty/bergeson/physics571/notes/L27spectnotation.pdf}{\textbf{article}} provides a good overview of the notation, in case you need a refresher. A hydrogen atom is in the n=3, l=2 state. Note that there are 2 possible values for J. 

    \item Draw an energy level diagram for helium. Show the transitions that produce the red and yellow lines. Note the differences in the structure and splitting between the energy levels for hydrogen and helium.

    \item Draw an energy level diagram illustrating the Zeeman effect for the red line of helium. When a 1 Tesla magnetic field is applied to helium, what happens to the energy levels and transitions that produce the red line of helium?

    \item What gives a spectral line a non-zero width? Estimate the line width for the first Balmer series. Assume that the pressure inside the tube is 5 torr and the temperature 600K, and that the lifetimes of the states are about 10ns.

    \item Draw a sketch of the diffraction grating spectrometer showing the placement of the optical elements and the path and focusing of a light beam as it goes from the source through the spectrometer and to the film or photomultiplier. Explain how the grating works. Calculate a representative value for the resolving power of the grating.

    \item Is the photomultiplier output current proportional to the number of incident photons, independent of wavelength? If not, how does the output depend on wavelength?

    \item Draw a sketch of the Fabry-Perot interferometer showing the placement of the optical elements and the path of a light beam as it goes from the source through the interferometer and through the telescope. Calculate the resolving power of the interferometer. Why is it necessary to use the interferometer instead of the grating for observing the Zeeman Effect?
       \\[36pt]
\end{enumerate}

\prelabsignatures

\midlab

\begin{enumerate}

    \item On day 3 of this lab, you should have successfully produced a plot of the Balmer-series lines, and made an estimate of the Rydberg constant. Show them to an instructor and ask for a signature. 
\\[36pt]
\end{enumerate}

\midlabsignatures{3}

\begin{enumerate}

\item On day 5 of this lab, you should have successfully observed the Zeeman splitting of the helium lines and estimated a value for the Bohr magneton. Demonstrate this to an instructor and ask for a signature.
\\[36pt]
\end{enumerate}

\midlabsignatures{5}

\checkpointsection 

\begin{enumerate}

\item \checkp{Preparation}

\item \checkp{Peak Finding}

\item \checkp{Additional Questions}

\item \checkp{Zeeman Picture}

\item \checkp{Zeeman Splitting}

\end{enumerate}

\end{document}