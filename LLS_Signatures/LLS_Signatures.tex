\documentclass{../signatures}
\labacronym{LLS}
\labtitle{Low Light Signal Measurements}

\begin{document}

\maketitle

\names

\prelab

\begin{enumerate}

    \item What is the purpose of having the SR760 FFT Spectrum Analyzer? What mathematical operation does it perform on an input signal? Give several examples of what goes in and what comes out.
    
    \item Why don’t we just put a filter in front of our photodiode which only passes light at the frequency given off by the LED (e.g. 635 nm)?

    \item Explain how chopping the steady-state signal from the LED helps you to recover the signal from the LED.
       \\[36pt]
\end{enumerate}

\prelabsignatures

\midlab

\begin{enumerate}

    \item Before beginning Experiment X, you should have produced a plot of the Johnson noise signal using the measurements obtained in Experiment IX.
\\[36pt]
\end{enumerate}

\midlabsignaturesLLS
\newpage
\checkpointsection 
\\[12pt]
\begin{enumerate}

\item \checkp{Fourier Components}

\item \checkp{Slope/Oct and Time Constant }

\item \checkp{1/f Noise}

\item \checkp{Capacitive Noise}

\item \checkp{Johnson-Noise Measurement}

\end{enumerate}

\end{document}