\documentclass{../signatures}
\labacronym{OPT}
\labtitle{Optical Pumping}

\begin{document}

\maketitle

\names

\prelab

\begin{enumerate}

    \item What is the general principle of optical pumping? Go over your derivation of the Breit-Rabi formula and the values of the Lande g-factors of the hyperfine energy levels of 85Rb and 87Rb. Draw qualitative energy-level diagrams for 85Rb and 87Rb showing the fine, hyperfine, and Zeeman splittings. How do the Lande g-factors affect the ordering of the Zeman levels? Show the transitions between these levels that are important to this experiment. Include these drawings in your write-up. For our rubidium system, what is the pumping process? Where is the pumped level? Where is the RF transition?
    
    \item Why do we modulate (vary sinusoidally) the external magnetic field? How would we take data if the magnetic field were not modulated?

    \item In this experiment, how will you determine the resonance frequency? How can you best estimate the error? Will the modulation amplitude affect your result? What data will you take, and what plots will you make?
       \\[36pt]
\end{enumerate}

\prelabsignatures

\midlab

\begin{enumerate}

    \item On day 2 of this lab, you should have successfully produced a plot of frequency versus current for at least one rubidium isotope, and have made an estimate of the earth’s magnetic field. Show them to an instructor and ask for a signature.
\\[36pt]
\end{enumerate}

\midlabsignatures{2}

\checkpointsection 

\begin{enumerate}

\item \checkp{DS345 Preparation}

\item \checkp{Resonance Conditions and Symmetry}

\item \checkp{Error Analysis Methods}

\end{enumerate}

\end{document}