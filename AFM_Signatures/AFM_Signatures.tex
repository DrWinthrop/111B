\documentclass{../signatures}
\labacronym{AFM}
\labtitle{Atomic Force Microscope}

\begin{document}

\maketitle

\names

\prelab

\begin{enumerate}

    \item How does atomic force microscopy differ from other forms of microscopy (e.g. SEM or optical)?  What are advantages? Disadvantages?
    
    \item What is a piezoelectric transducer? Where are they used in our AFM, and why are they used? 

    \item Draw, and briefly explain the different features of the typical force distance curve for a material. Be sure to note the attractive and repulsive regions, why these different regions exist and which AFM modes operate in which regions. 

    \item Explain how an AFM takes a scan of a sample.  How does the AFM produce a 3D image? You should explain the laser and detector, cantilever, tip, feedback loop, and z piezo. Also explain the differences between non-vibrating (constant force, and constant height) mode and vibrating mode. Identify which modes you will use in this lab and when you will use them. 

    \item Explain what the important parameters do in the Scan Setup, Topo Scan, and System tab (frequency select, manual z motor control, automated tip approach, range check, scan lines, scan size, rotation, Z feedback, display, HV Z gain, XY parameters, tip approach parameters, calibration)

    \item Give examples of two scenarios where you are likely to break a tip. BE VERY CAREFUL.
       \\[36pt]
\end{enumerate}

\prelabsignatures

\midlab

On day 2 of this lab, you should have the ability to scan samples accurately and reliably. Show an image processed topo scan of the reference slide to an instructor.  Take a profile of a row of features to show that your measurements and calibrations are correct. How close are your measurements compared with the datasheet?

\begin{enumerate}

    \item What is a scanned image’s orientation relative to the camera’s image orientation?
    
    \item Which direction does the scanned image move if you move the physical sample to the left? What about if you move the physical sample up?
\\[36pt]
\end{enumerate}

\midlabsignatures{2}

\checkpointsection 

\begin{enumerate}

\item \checkp{Laser Alignment}

\item \checkp{Tip Settings}

\item \checkp{Chosen Experiment}

\item \checkp{Noise Floor}

\item \checkp{Boltzmann's Constant}

\end{enumerate}

\end{document}