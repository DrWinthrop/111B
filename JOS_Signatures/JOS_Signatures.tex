\documentclass{../signatures}
\labacronym{JOS}
\labtitle{Josephson Junction}

\begin{document}

\maketitle

\names

\prelab

\begin{enumerate}

    \item What is a Josephson–junction?
    
    \item What is a Cooper–pair?

    \item How does the DC Josephson effect work? What is the AC Josephson effect? How are they useful?

    \item Why is it important to know the number 2e/h; what does it mean?

    \item How do you construct the Josephson–junction used in this experiment? (Hint 4 wire junction)

    \item Explain how you remove the junction assembly from the probe and put it back in.
       \\[36pt]
\end{enumerate}

\prelabsignatures

\midlab

\begin{enumerate}

    \item On day 3 of this lab, show your photo of the DC effect with calibrated axis and photo of the AC effect, with calibrated axis, to a GSI. Also you present your measured corrected value for 2$e/h$ in units of MHz/$\mu$V (2$e/h= 4.83593718 \times 1014$), with uncertainty, to an instructor and ask for a signature.
\\[36pt]
\end{enumerate}

\midlabsignatures{3}

\checkpointsection 

\begin{enumerate}

\item \checkp{Low-Frequency Oscillator}

\item \checkp{Steps}

\item \checkp{DC Effect}

\item \checkp{AC Effect}

\item \checkp{Precise Measurement of the RF Frequency}

\end{enumerate}

\end{document}